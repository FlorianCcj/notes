% ##########################################################################################################
% Param
% ##########################################################################################################

\documentclass[10pt, a4paper]{moderncv}
\moderncvtheme[blue]{classic}
\usepackage[utf8]{inputenc}
\usepackage[inline]{enumitem}

% Marge aux 4 coins de la page, ici elles sont réduites pour gagner de la place
\usepackage[
  top=1.0cm, bottom=1.0cm,
  left=1.0cm, right=1.0cm
]{geometry}
% Largeur de la colonne de gauche pour les dates
\setlength{\hintscolumnwidth}{2.7cm}

% remove step between item
\usepackage{enumitem}
\setlist{nolistsep,leftmargin=*}

% ##########################################################################################################
% Whoami
% ##########################################################################################################

\firstname{Florian}
\familyname{ccj}
\email{askme.com}
\mobile{+33 (7) 66 66 66 66}
\address{Bastille}{75011 Paris}
%\homepage{}
%\extrainfo{}

\title{Administrateur système / Devops}

% ##########################################################################################################
% Content
% ##########################################################################################################

\begin{document}
\maketitle
% Marge négative entre le titre et la partie expérience, pour gagner de la place
\vspace*{-2.5\baselineskip}

% **********************************************************************************************************
% Experience
% **********************************************************************************************************

\section{Expériences}

\subsection{Professionnelle}

\cventry{oct. 2018\\à Aujourd'hui}{Admin. Sys.}{Ministère des armées}{Paris}{}{%
  \begin{itemize}%
    \item Mise à disposition d'outil pour les développeurs (Gitlab + CI, SonarQube, Artifactory) ;
    \item Rapatriement de dépot pour permettre de travailler hors connection ;
    \item Maintient en condition opérationnel et supervision des outils mis à disposition ;
    \item Accompagnement des développeurs dans la mise en place d'intégration et de déploiement continue ;
    \item Sensibilisation et accompagnement des développeurs dans la mise en place de processus de qualité et de sécurisation des développements ;
    \item Mise en place de communauté de pratique de développement et de pratique de qualité ;
    \item Exploitation de cluster Kubernetes.
  \end{itemize}
}
\cventry{oct. 2016\\à oct. 2018}{Developpeur}{Ministère de la Défense}{Paris}{}{%
  \begin{itemize}%
    \item Développement en Php/Symfony 2/3,
      \begin{itemize}%
        \item Site web complet grace a Twig ;
        \item Interaction avec une base MySQL au travers de l'ORM Doctrine ;
        \item Mise à disposition d'API ;
        \item Interaction avec d'autres applications par API.
      \end{itemize}%
    \item Développement en Angular 2/4.
  \end{itemize}
}
\cventry{2015\\(3 mois)}{Stage, Développeur de job de traitement de log}{Tagattitude (http://en.tagpay.fr)}{Rocquencourt}{}{%
  \begin{itemize}%
    \item Developpement en Php/MySQL ;
    \item Parcours de log, pour faire des statistiques.
  \end{itemize}
}
%\cventry{2012\\(1 mois)}{Stage, Développeur Script de reprise de donnée}{(http://www.magelec.com)}{Paris}{}{%
%  \begin{itemize}%
%    \item Développement en Php/MySQL ;
%    \item Extraction de fichiers excel en CSV fournis par le fournisseur pour allimenter la base de donnée des articles mis à disposition sur le site.
%  \end{itemize}
%}

\subsection{Associative}

\cventry{2019 --- Aujourd'hui}{L'Aube d'Ailleurs}{Association de jeu}{Nantes}{France}{%
  \begin{itemize}
    \item Organisation dun Jeu de Role Grandeur Nature, politique, pour une quarantaine de personne.
  \end{itemize}
}
\cventry{Sept 2011\\à Sept 2016}{l'Appel de Sussu}{Association de jeu de société}{Ivry}{France}{%
  \begin{itemize}
    \item Organisation d'évènement autour du thème du jeu ;
    \item Présidence de l'association (2015--1016).
  \end{itemize}
}
\cventry{Sept 2000\\à Sept 2017}{EEDF (scoutisme)}{}{Paris}{France}{%
  \begin{itemize}
    \item Encadrement d'enfant entre 8 et 15 ans ;
    \item Organisation de projet sur l'année, pour un aboutissement durant les vacances d'été ;
    \item Journalisme et participation à un rassemblement mondial (Jambore 7--8/2011 en Suède).
  \end{itemize}
}

% **********************************************************************************************************
% Formation
% **********************************************************************************************************

\section{Formations}

\cventry{2011 --- 2016}{ESME Sudria}{Ecole d'Ingénieur}{Ivry-Sur-Seine}{Majeur Informatique, Spécialité Technologie Emergente}{%
  \begin{itemize}%
    \item Projet master 2 en \textbf{NodeJS}: Reconnaissance vocale et apprentissage autonome dans la gestion des peripherique au quotidien (2015--2016 7 mois)
    \item Projet master 1 en \textbf{NodeJS}: Synchronisation de deux dossiers dans le cadre de gestion de données (2014--2015 4 mois)
%    \item \textbf{C++}: Programme générant des deplacements dimultannés et aléatoire de piece d'échequier en respectant les règles du jeu (2015 2 mois)
  \end{itemize}
}
\cventry{2014 (5 mois)}{Shanghai JioTong University}{Ecole d'ingénieur}{Shanghai}{}{(étude de système embarqués, base de données, c++)}

% **********************************************************************************************************
% Formation
% **********************************************************************************************************

%\section{Certifications}%

%\cventry{Juillet 2017}{Certification LFCE}{The Linux Foundation}{Valide 2 ans}{\href{https://www.alexis-madrzejewski.com/ressources/lfce-madrzejewski.pdf}{LFCE-1700-0334-0200}}{%
%Certification \textit{Linux Foundation Certified Engineer} créé par \textit{The Linux Foundation}.
%Les compétences testées sont en rapport avec Linux et l'administration système avancé.\newline
%}
%\cventry{Août 2016}{Certification LFCS}{The Linux Foundation}{Valide 2 ans}{\href{https://www.alexis-madrzejewski.com/ressources/lfcs-madrzejewski.pdf}{LFCS-1600-0922-0100}}{%
%Certification \textit{Linux Foundation Certified System Administrator}.
%}

% **********************************************************************************************************
% Competence
% **********************************************************************************************************

\section{Compétences} % \section{Compétences en informatique}
\cvitem{\textbf{OS}}{Centos 7, Debian 7/8/9}
\cvitem{\textbf{Prog.}}{CSS, HTML, PHP, SQL, Python3, BASH, JS, Docker}
\cvitem{\textbf{Framework/Lib.}}{Angular 2/4, Symfony 2/3/4, Sphynx, Bootstrap 3}
\cvitem{\textbf{Admin.}}{Ansible, Apache, Nginx, K8S, Artifactory, Vault, SonarQube, MySQL, Git, Gitlab, Gitlab-ci}
%\cvitem{\textbf{Sécurité}}{Notions SELinux, CSF, iptables, firewalld}
\cvlanguage{\textbf{Anglais}}{lu, écrit, parlé --- \underline{Technique}: lu, écrit }{Toeic 815 \texttt{/} 990}
\cvdoubleitem{\textbf{Espagnol}}{Niveau Lycée}{\textbf{Chinois}}{Notion}

% **********************************************************************************************************
% Interests
% **********************************************************************************************************

\section{Centres d'intérêt}

\cvdoubleitem{\textbf{Jeu}}{Jeu de société, Go, Jeu de role}{\textbf{Lecture}}{Manga, Fantasy, Science Fiction}
\cvitem{\textbf{Art}}{Cirque, Cardistry, Magie, Dessin, Couture, Cuisine, Flute, Web Design}
\cvitem{\textbf{Science}}{Mathématique, Informatique}

\end{document}
